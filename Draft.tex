\documentclass[12 pt,a4paper]{amsart}


\usepackage{amsmath,amsfonts,amssymb,verbatim}
\usepackage{graphicx}
\usepackage{xcolor}

\usepackage{setspace}
\singlespacing
\usepackage{geometry}
\geometry{hmargin=2.cm,vmargin=2.cm}

\title{Analyse numérique d'un modèle de type Bloch non linéaire }

\author{Kolé KEITA}
\author{Okpo Dorgeles Okaingny}


\begin{document}
\maketitle


\bigskip

Abstract:


\vspace{0.2cm}
\noindent Keywords : 


\section{Introduction}
\begin{itemize}
\item \textcolor{red}{Contexte général sur les modèles de type Bloch}
\item \textcolor{red}{Souligner la raison et l'utilité de ramener le modèle de type Bloch à variables réelles}
\item \textcolor{red}{Parler des études mathématiques effectuées sur le modèle à variables réelles}
\item \textcolor{red}{ Présenter la nécessité de faire l'analyse numérique}
\end{itemize}


\section{Dérivation du modèle de type Bloch non linéaire à variables réelles }
\begin{itemize}
\item \textcolor{red}{Présenter le modèle de type Bloch avec le champ électrique, la relaxation de Pauli et l'interaction de Coulomb (référer à la thèse et à l'article publié avec Brigitte)}
\item \textcolor{red}{Énumérer les propriétés qualitatives vérifiées et parler du problème bien posé  (référer à la thèse et à l'article publié avec Brigitte)}
\item \textcolor{red}{Écrire le modèle à trois niveaux en vérifiant les propriétés dans les cas sans intra-bandes et sans inter-bandes}
\item \textcolor{red}{ Présentation du modèle à variables réelles pour un système à trois niveaux}
\end{itemize}


\section{Analyse du modèle à variables réelles }
\begin{itemize}
\item \textcolor{red}{ Vérifier les propriétés qualificatives et le caractère bien posé du nouveau modèle}
\item  \textcolor{red}{ Analyser les modèles sans intra-bandes et sans inter-bandes}
\end{itemize}



\section{Discrétisation du modèle à variable réelle}
\begin{itemize}
\item \textcolor{red}{Méthodes classiques : Euler, Runge-Kutta, Crank-Nicolson}
\item \textcolor{red}{Vérifier les propriétés des solutions numériques obtenues avec ces méthodes classiques et la stabilité numérique}
\item \textcolor{red}{Méthodes de Splitting et vérification des propriétés} 
\end{itemize}

\section{Simulation numérique}
\textcolor{red}{Ajouter les simulations}


\section{Conclusion}
\end{document}
